\documentclass{article}
\usepackage{graphicx} % Required for inserting images
\usepackage[T2A]{fontenc}
\usepackage[utf8]{inputenc}
\usepackage[english, russian]{babel}
\usepackage{amsfonts}
\usepackage{amsmath}
\usepackage[left=2cm,right=2cm,
    top=2cm,bottom=2cm,bindingoffset=0cm]{geometry}
\setlength\parindent{1.5em}
\DeclareMathOperator{\grad}{grad}
\DeclareMathOperator{\lagr}{\mathcal{L}}


\title{Домашнее задание 2 (тервер)}
\author{Андрей Зотов}
\date{Сентябрь 2023}

\begin{document}

\maketitle

\section*{Задача 1}
{\bf Ответ: } a) $P(X=2)=0.5$; б) \begin{tabular}{|c|c|c|c|}
     \hline
     $\textbf{X}^2$ & 0 & 1 & 4\\
     \hline
     \textbf{P} & 0.15 & 0.3 & 0.55\\
     \hline
\end{tabular}.
\\
\\
{\bf Решение.}
\par
a) Т.к. $\sum\limits_{k=-2}^2 P(X=k) =1$, то $P(X=2)=1-\sum\limits_{k=-2}^1 P(X=k)=1-(0.05+0.1+0.15+0.2)=0.5$.
\par
б) Т.к. величина $X$ принимает значения из множества $\{-2,-1,0,1,2\}$, то величина $X^2$ принимает значения из множества $\{0, 1, 4\}$, при этом $P(X^2=0)=P(X=0)=0.15;\ P(X^2=1)=P(X=-1)+P(X=1)=0.1+0.2=0.3;\ P(X^2=4)=P(X=-2)+P(X=2)=0.05+0.5=0.55$.
\section*{Задача 2}
{\bf Ответ: } $\frac{2}{3}$.
\\
\\
{\bf Решение.} 
\par
Событие $A=$<<решка первый раз выпала на нечетном броске>> состоит из следующих элементарных исходов: 
$\\
\omega_0=\text{<<первый раз решка выпала на первом испытании>>};\\
\omega_1=\text{<<первый раз решка выпала на третьем испытании>>};\\
\text{...}\\
\omega_k=\text{<<первый раз решка выпала на $2k+1$-ом испытании>>};\\
\text{...}\\$

Т.е. $A=\bigsqcup\limits_{k=0}^\infty \omega_k$, поэтому $P(A)=\sum\limits_{k=0}^\infty P(\omega_k)$. Считаем монету симметричной (т.е. вероятность выпадения решки $\frac{1}{2}$) и испытания независимыми в совокупности. Событие $\omega_k$ означает выпадение орлов в первых $2k$ испытаниях и затем выпадение решки в последнем $2k+1$-ом испытании, поэтому $P(\omega_k)=\left(\frac{1}{2}\right)^{2k}\frac{1}{2}=\frac{1}{2}\cdot\left(\frac{1}{4}\right)^k$.
\par
Таким образом $P(A)=\frac{1}{2}\sum\limits_{k=0}^\infty\left(\frac{1}{4}\right)^k=<\text{пользуемся формулой для суммы геом. прогрессии}>=\frac{1}{2}\cdot\frac{4}{3}=\frac{2}{3}$.
\section*{Задача 3}
{\bf Ответ: } a) $\frac{1}{8}$; б) $0$; в) $0$; г) $\frac{1}{4}$; д) $\frac{3}{8}$.
\\
\\
{\bf Решение.}
\par
a) $P(X=1)=P(X \leq 1) - P(X < 1) = F_X(1)-\lim\limits_{x\to 1-0}F_X(x)=F_X(1)-F_X(1-0)=\frac{1}{4}-\frac{1}{8}=\frac{1}{8}$;
\par
б) $P(X=2)=P(X \leq 2) - P(X < 2) = F_X(2)-F_X(2-0)=\frac{1}{2}-\frac{1}{2}=0$;
\par
в) $P(X \in (1.5; 2]) = P(X \leq 2) - P(X \leq 1.5) = F_X(2)-F_X(1.5)=\frac{1}{2}-\frac{1}{2}=0$;
\par
г) $P(X \in (1; 2]) = P(X \leq 2) - P(X \leq 1) = F_X(2) - F_X(1) = \frac{1}{2} - \frac{1}{4}=\frac{1}{4}$;
\par
д) $P(X \in [1; 2])=P(X=1) + P(X \in (1; 2]) = \frac{1}{8} + \frac{1}{4} = \frac{3}{8}$.
\section*{Задача 4}
{\bf Ответ: } a) В частности возможна плотность нормального распределения с любыми параметрами (параметры зависят от соотношения числа сайтов $N$ и числа ссылок $M$) б) $F_X(5)=1-\frac{1}{5^{1.1}}\approx 0.83$.
 \\
 \\
{\bf Решение.}
 \par
 Пусть $X$ - случайная величина <<число ссылок на случайном выбранном сайте>> (исходящая степень случайной вершины в веб-графе), а $Y$ - <<число ссылок на случайно выбранный сайт>> (входящая степень случайной вершины в веб-графе).
 \par
 a) Плотность случайной величины $X$, распределенной согласно распределению Парето с параметрами $x_m=1,\ k=1.1$, будет $f_X(x)=
 \begin{cases}
    \frac{kx_m^k}{x^{k+1}}, &x \geq x_m\\
    0, &x < x_m
 \end{cases} =
 \begin{cases}
   \frac{1.1}{x^{2.1}}, &x \geq 1\\
   0,& x < 1
 \end{cases}$.
 Требуется найти плотность распределения величины $Y$ - $f_Y(x)$.
 \par
 Представляется, что характер распределения $Y$ не зависит от распределения $X$ - по крайней мере из условия это никак не следует. В частности, можно показать, что плотность распределения $Y$ может быть нормальной.
 \par
 Допустим, что ссылки указывают на любой сайт (в т.ч. числе и на себя) равновероятно. Тогда, если $N$ - это число всех сайтов, а $M$ - число всех ссылок, то вероятность, что на случайный сайт будет указывать ровно $k$ ссылок будет $P(Y=k)=\binom{M}{k} \cdot\frac{1}{N^k}\cdot\left(\frac{N-1}{N}\right)^{M-k}$ (биномиальное распределение). Т.е. имеем дискретное распределение, у которого вообще говоря нет плотности. Но можно рассмотреть предельный случай биномиального распределения, которое как известно является нормальным распределением с известной функцией плотности.
 \par
 Пусть $Y_n$ — это число ссылок на случайный сайт при условии, что всего ссылок $n=M$. Пусть $p=\frac{1}{N}$ и $q=1-p$. Тогда согласно предельной теореме Муавра-Лапласа для фиксированных $z_1$ и $z_2$ при $n\to\infty$ имеем:
 $$P(np+z_1\sqrt{npq}\leq Y_n\leq np+z_2\sqrt{npq})\to \mathfrak{N}(z_2)-\mathfrak{N}(z_1)$$
 где $\mathfrak{N}(x)=\frac{1}{\sqrt{2\pi}}\int\limits_{-\infty}^x e^{-y^2/2}\,dy$ и величины $np$ и $npq$ сохраняют устойчивость, т.е. имеют конечные пределы. 
 \par
 Отсюда видно, что если, например, величина $np=n/N$ близка к нулю при $n\to\infty$ (т.е. $N$ тоже неограниченно растет в пределе), а величина $npq=n(N-1)/N^2$ близка к 1 при $n\to\infty$, то $P(Y_n \leq x)\to F_Y(x)=\mathfrak{N}(x)$, т.е. плотность $f_Y(x)=\frac{1}{\sqrt{2\pi}}e^{-x^2/2}$.
 \par
 Вообще говоря, в пределе можно получить плотность нормального распределения с любыми параметрами - зависит от структуры веб-графа и от асимптотического поведения величин $M/N, M(N-1)/N^2$.
 \par
 б) В данном случае надо найти $F_X(5)$. Функция распределения случайной величины $X$ имеет вид $F_X(x)=
 \begin{cases}
    1-\left(\frac{x_m}{x}\right)^{k}, &x \geq x_m\\
    0, &x < x_m
 \end{cases}=
 \begin{cases}
    1-\frac{1}{x^{1.1}}, &x\geq 1\\
    0, &x < 1
 \end{cases}$, поэтому искомая вероятность будет $F_X(5)=1-\frac{1}{5^{1.1}}\approx0.83$.
\section*{Задача 5}
{\bf Ответ: } $\frac{9}{16}$.
\\
\\
{\bf Решение.} 
\par
Т.к. функция плотности $f_X(x)$ определена для всех точек отрезка $[\frac{1}{2}; 2]$, то функция распределения $F_X(x)$ непрерывна на этом отрезке и поэтому $P(X \in [\frac{1}{2}; 2])=\int\limits_{\frac{1}{2}}^2 f_X(x)\,dx$. Этот интеграл - это площадь под графиком $f_X(x)$ на отрезке $[\frac{1}{2}; 2]$, т.е. площадь $S$ трапеции с основаниями $a=\frac{1}{4},\ b=\frac{1}{2}$ и высотой $h=\frac{3}{2}$, т.е.  $P(X \in [\frac{1}{2}; 2])=S=\frac{1}{2}(a+b)h=\frac{1}{2}\cdot\frac{3}{4}\cdot\frac{3}{2}=\frac{9}{16}$.
\section*{Задача 6}
{\bf Ответ: } a) $\lambda=-\frac{1}{5}\ln \frac{2}{5}\approx 0.18$; б) $\frac{21}{25}=0.84$.
\\
\\
{\bf Решение.}
\par
a) Функция экспоненциального распределения случайной величины $X$ имеет вид $F_X(x)=
\begin{cases}
    1-e^{-\lambda x}, &x \geq 0\\
    0,                &x < 0
\end{cases}$. При этом известно, что $P(X > 5)=\frac{2}{5}$, т.е. $P(X > 5)=1 - P(X \leq 5)=1-F_X(5)=1-(1-e^{-5\lambda})=\frac{2}{5}\Leftrightarrow e^{-5\lambda}=\frac{2}{5}\Leftrightarrow -5\lambda = \ln\frac{2}{5}\Leftrightarrow \lambda=-\frac{1}{5}\ln\frac{2}{5}\approx 0.18$.
\par
б) Вероятность того, что разговор продлится не более 10 минут будет $P(X \leq 10) = F_X(10)=1-e^{-10\lambda}=1-e^{2\ln\frac{2}{5}}=1-e^{\ln\frac{4}{25}}=1-\frac{4}{25}=\frac{21}{25}=0.84$.
\section*{Задача 7}
{\bf Ответ: } a) $Y\sim U[2,4]$; б) $F_Z(x)=\begin{cases}1,&x \geq \ln 2\\e^x-1,&x\in [0, \ln 2)\\0, &x < 0\end{cases}$; $f_Z(x)=\begin{cases}e^x,&x\in [0, \ln 2]\\0,&x \notin [0, \ln 2]\end{cases}$.
\\
\\
{\bf Решение.}
\par
Пусть $X\sim U[0,1]$, тогда $f_X(x)=\begin{cases}1,&x\in[0;1]\\0,&x\notin[0;1]\end{cases}$ и $F_X(x)=\begin{cases}1,&x\geq1\\x,&x\in[0;1)\\0,&x<0\end{cases}$.
\par
a) Если $Y=(X+1)\cdot 2$, то функция распределения $F_Y(x)=P(Y\leq x)=P((X+1)\cdot 2 \leq x)=P(X\leq \frac{x-2}{2})=F_X(\frac{x-2}{2})=\begin{cases}1,&\frac{x-2}{2}\geq 1\\\frac{x-2}{2},&\frac{x-2}{2}\in[0;1)\\0,&\frac{x-2}{2} < 0\end{cases}=\begin{cases}1,&x \geq 4\\\frac{x-2}{2},&x\in[2;4)\\0,&x < 2\end{cases}$, т.е. $Y\sim U[2;4]$.
\par
б) Если $Z=\ln(X+1)$, то функция распределения $F_Z(x)=P(Z \leq x)=P(\ln(X+1) \leq x)=P(X \leq e^x-1)=F_X(e^x-1)=\begin{cases}1,&e^x-1\geq 1\\e^x-1,&e^x-1 \in [0;1)\\0,&e^x-1 < 0\end{cases}=\begin{cases}1,&x \geq \ln 2\\e^x-1,&x\in [0, \ln 2)\\0, &x < 0\end{cases} \Rightarrow \text{плотность} f_Z(x)=\begin{cases}e^x,&x\in [0, \ln 2]\\0,&x \notin [0, \ln 2]\end{cases}$.
\end{document}
