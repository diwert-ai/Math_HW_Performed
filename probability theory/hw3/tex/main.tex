\documentclass{article}
\usepackage{graphicx} % Required for inserting images
\usepackage[T2A]{fontenc}
\usepackage[utf8]{inputenc}
\usepackage[english, russian]{babel}
\usepackage{amsfonts}
\usepackage{amsmath}
\usepackage[left=2cm,right=2cm,
    top=2cm,bottom=2cm,bindingoffset=0cm]{geometry}
\setlength\parindent{1.5em}


\title{Домашнее задание 3 (тервер)}
\author{Андрей Зотов}
\date{Сентябрь 2023}

\begin{document}

\maketitle

\section*{Задача 1}
{\bf Ответ: } $P(S_1=i|S_2=0)=\begin{cases}0,&i=10,11,\dots,18;\\\frac{2}{19},&i=1,2,\dots,9;\\\frac{1}{19},&i=0.\end{cases}$
\\
\\
{\bf Решение.}
\par
В данной задаче пространство $\Omega=\{00,01,\dots,99\}$. Считаем все элементарные исходы равновероятными, т.е. $\forall \omega\in\Omega:\ P(\omega)=\frac{1}{|\Omega|}=\frac{1}{100}$.
По определению условной вероятности:
$$P(S_1=i|S_0=0)=\frac{P(S_1=i,S_0=0)}{P(S_0=0)}$$
\par
Событие $S_0=0$ содержит 19 элементарных исходов (10 чисел, начинающихся с 0 и 9 чисел нулем не начинающихся, но нулем заканчивающихся). Поэтому $P(S_0=0)=\frac{19}{100}$.
\par
Событие $S_1=i,S_0=0$ содержит разное кол-во элементарных исходов в зависимости от значения $i$. Рассмотрим три случая:
\par
1. $i=0$. В этом случае событие $S_1=0,S_0=0$ возможно только при одном элементарном исходе - $00$, поэтому $P(S_1=0,S_0=0)=\frac{1}{100}$.
\par
2. $i \in \{1, 2,\dots,9\}$.  В этом случае событие $S_1=i,S_2=0$ возможно только при двух элементарных исходах $0i$ и $i0$, поэтому $P(S_1=i,S_0=0)=\frac{2}{100}$.
\par
3. $i \in \{10, 11,\dots,18\}$. В этом случае событие $S_1=i,S_2=0$ пустое (не содержит элементарных исходов, т.к. $\max\limits_{S_0=0} S_1 = 9$), поэтому $P(S_1=i,S_0=0)=0$.
\par
Таким образом:
$$P(S_1=i|S_0=0)=\frac{P(S_1=i,S_0=0)}{P(S_0=0)}=\begin{cases}0,&i=10,11,\dots,18;\\\frac{2}{19},&i=1,2,\dots,9;\\\frac{1}{19},&i=0.\end{cases}$$
\section*{Задача 2}
{\bf Ответ: } $\frac{67}{105}\approx 0.64$.
\\
\\
{\bf Решение.}
\par
Введем следующие события:
\begin{itemize}
    \item $X=\{\text{Из 3-й урны вытащили черный шар}\}$;
    \item $H_0=\{\text{Из 1-й урны вытащили белый шар и из 2-й урны вытащили белый шар}\}$;
    \item $H_1=\{\text{Из 1-й урны вытащили белый шар и из 2-й урны вытащили черный шар}\}$;
    \item $H_2=\{\text{Из 1-й урны вытащили черный шар и из 2-й урны вытащили белый шар}\}$;
    \item $H_3=\{\text{Из 1-й урны вытащили черный шар и из 2-й урны вытащили черный шар}\}$.
\end{itemize}
\par
Т.к. $\Omega=H_0\sqcup H_1\sqcup H_2\sqcup H_3$, то $P(X)=P(X|H_0)P(H_0)+P(X|H_1)P(H_1)+P(X|H_2)P(H_2)+P(X|H_3)P(H_3)$, где:
\begin{itemize}
    \item $P(H_0)=\frac{1}{10}\cdot\frac{5}{6}=\frac{5}{60}$;
    \item $P(H_1)=\frac{1}{10}\cdot\frac{1}{6}=\frac{1}{60}$;
    \item $P(H_2)=\frac{9}{10}\cdot\frac{5}{6}=\frac{45}{60}$;
    \item $P(H_3)=\frac{9}{10}\cdot\frac{1}{6}=\frac{9}{60}$.
\end{itemize}
(тут считаем, что доставание шара из 1-й урны не зависит от доставания шара из 2-й урны и наоборот, т.е. вероятности можно перемножить)
\par
Если произошло событие $H_0$, то в 3-й урне будет 10 черных шаров и 4 белых, поэтому условная вероятность достать черный шар будет $P(X|H_0)=\frac{10}{14}$. Аналогично получаем $P(X|H_1)=\frac{9}{14},\ P(X|H_2)=\frac{9}{14},\ P(X|H_3)=\frac{8}{14}$.
\par
Таким образом:
$$P(X)=P(X|H_0)P(H_0)+P(X|H_1)P(H_1)+P(X|H_2)P(H_2)+P(X|H_3)P(H_3)=\frac{5}{60}\cdot\frac{10}{14}+\frac{1}{60}\cdot\frac{9}{14}+\frac{45}{60}\cdot\frac{9}{14}+\frac{9}{60}\cdot\frac{8}{14}=\frac{67}{105}.$$
\section*{Задача 3}
{\bf Ответ: } Работа скорее всего принадлежит первому студенту. Вероятность того, что она принадлежит первому студенту $\frac{27}{46}\approx 0.59$.
\\
\\
{\bf Решение.}
\par
Если $i$-й студент решает любую задачу с вероятностью $p_i$ и при этом решение задач не зависит друг от друга, то вероятность того, что $i$-й студент решит ровно 3 задачи из 4-х будет (согласно биномиальному закону) $P_i=\frac{4!}{3!1!}\cdot p_i^3\cdot(1-p_i)=4p_i^3(1-p_i)$. Поэтому имеем:
\begin{itemize}
    \item $P_1 = 4\cdot\frac{3^3}{4^3}\cdot\frac{1}{4}=\frac{27}{64}$;
    \item $P_2 = 4\cdot\frac{1^3}{2^3}\cdot\frac{1}{2}=\frac{16}{64}$;
    \item $P_3 = 4\cdot\frac{1^3}{4^3}\cdot\frac{3}{4}=\frac{3}{64}$.
\end{itemize}
\par
Т.е. работа скорее всего принадлежит первому студенту. Найдем вероятность этого события.
\par
Пусть событие $X_i$ <<Работа принадлежит $i$-му студенту>> и событие $H$ <<Решено 3 задачи>>. Тогда требуется найти вероятность:
$$P(X_1|H)=\frac{P(X_1 H)}{P(H)}=\frac{P(H|X_1)P(X_1)}{P(H)}.$$
\par
Т.к. $\Omega=X_1\sqcup X_2\sqcup X_3$, то 
$$P(H)=P(H|X_1)P(X_1)+P(H|X_2)P(X_2)+P(H|X_3)P(X_3).$$
\par
Считаем, что преподаватель мог равновероятно получить работу от любого студента, поэтому $P(X_i)=\frac{1}{3}$.
\par
Таким образом
$$P(X_1|H)=\frac{\frac{1}{3}P(H|X_1)}{\frac{1}{3}(P(H|X_1)+P(H|X_2)+P(H|X_3))}$$
\par
При этом событие $H|X_i$ означает, что решено 3 задачи при условии, что работа принадлежит $i$-му студенту, т.е. $P(H|X_i)=P_i$. Таким образом
$$P(X_1|H)=\frac{P_1}{P_1+P_2+P_3}=\frac{27}{27+16+3}=\frac{27}{46}\approx 0.59.$$
\section*{Задача 4}
{\bf Ответ: } При $N\in\{9, 10\}$. Если $N=9$, то распределение $X$ будет \begin{tabular}{|c|c|c|c|}
     \hline
     $\textbf{X}$ & 0 & 1 & 2\\
     \hline
     \textbf{P} & $\frac{3}{18}$ & $\frac{5}{9}$ & $\frac{5}{18}$\\
     \hline
\end{tabular}.
Если $N=10$, то распределение $X$ будет \begin{tabular}{|c|c|c|c|}
     \hline
     $\textbf{X}$ & 0 & 1 & 2\\
     \hline
     \textbf{P} & $\frac{2}{9}$ & $\frac{5}{9}$ & $\frac{2}{9}$\\
     \hline
\end{tabular}.
\\
\\
{\bf Решение.}
\par
Пусть событие $A_1$ <<Первая рыба из двух оказалась помеченная>> и событие $A_2$ <<Вторая рыба из двух оказалась помеченная>>, тогда (учитывая, что $A_1\bar A_2 \bar A_1 A_2=\emptyset$):
$$P(X=1)=P(A_1\bar A_2)+P(\bar A_1 A_2)=P(A_1)P(\bar A_2 |A_1)+P(\bar A_1)P(A_2|\bar A_1)$$
Где $P(A_1)=\frac{5}{N},\ P(\bar A_1)=\frac{N-5}{N}$. Если произошло событие $A_1$, это значит, что в озере $4$ помеченные рыбы и $N-5$ непомеченных, т.е. условная вероятность $P(\bar A_2 |A_1)=\frac{N-5}{N-1}$. Аналогично получаем, что $P(A_2|\bar A_1)=\frac{5}{N-1}$.
\par
Таким образом:
$$P(X=1)=\frac{5(N-5)}{N(N-1)}+\frac{5(N-5)}{N(N-1)}=\frac{10(N-5)}{N(N-1)}$$
При этом $N \geq 5$. 
\par
Найдем максимум функции $f(x)=\frac{x-5}{x(x-1)}$ при $x \geq 5$ (очевидно максимум функции $10f(x)$ достигается в той же точке, если этот максимум вообще существует). 
\par
Заметим, что:
$$f'(x)=\frac{x(x-1)-(2x-1)(x-5)}{x^2(x-1)^2}=-\frac{(x-5)^2-20}{x^2(x-1)^2}$$
При $x \geq 5$ знаменатель в ноль не обращается, поэтому искомый максимум удовлетворяет системе 
$$\begin{cases}(x-5)^2=20\\x \geq 5\end{cases}\Leftrightarrow x=5+2\sqrt{5}.$$
\par
Т.к. $2 < \sqrt{5} < 2.5$, то $9 < 5+2\sqrt{5} < 10$. При этом $f'(9) > 0$ и $f'(10) < 0$, т.е. единственный допустимый экстремум $x=5+2\sqrt{5}$ является максимумом. 
\par
Таким образом (учитывая, что $N$ целое) максимум $P(X=1)$ может достигаться либо при $N=9$, либо при $N=10$, либо в обоих случаях сразу. 
\par
При $N=9$ получаем $P(X=1)=\frac{10\cdot 4}{9\cdot 8}=\frac{5}{9}$. При $N=10$ получаем $P(X=1)=\frac{10\cdot 5}{10\cdot 9}=\frac{5}{9}$. Т.е. $P(X=1)$ достигает максимума $\frac{5}{9}\approx 0.56$ при $N\in\{9, 10\}$. 
\par
Случайная величина $X$ может принимать только три значения: $0,1,2$. Поэтому чтобы найти распределение $X$, нужно найти $P(X=0)$ и $P(X=2)$. Рассуждая также как в случае $P(X=1)$ получаем:
\begin{itemize}
    \item $P(X=0)=P(\bar A_1 \bar A_2)=P(\bar A_1)P(\bar A_2|\bar A_1)=\frac{N-5}{N}\cdot\frac{N-6}{N-1}$;
    \item $P(X=2)=P(A_1 A_2)=P(A_1)P(A_2|A_1)=\frac{5}{N}\cdot\frac{4}{N-1}$.    
\end{itemize}
\par
Таким образом если $N=9$, то распределение $X$ будет \begin{tabular}{|c|c|c|c|}
     \hline
     $\textbf{X}$ & 0 & 1 & 2\\
     \hline
     \textbf{P} & $\frac{3}{18}$ & $\frac{5}{9}$ & $\frac{5}{18}$\\
     \hline
\end{tabular}.
Если $N=10$, то распределение $X$ будет \begin{tabular}{|c|c|c|c|}
     \hline
     $\textbf{X}$ & 0 & 1 & 2\\
     \hline
     \textbf{P} & $\frac{2}{9}$ & $\frac{5}{9}$ & $\frac{2}{9}$\\
     \hline
\end{tabular}.
\section*{Задача 5}
{\bf Ответ: } a) 0.1; б) 0.15; в) 0.35; г) 0.05.
\\
\\
{\bf Решение.}
\par
a) $P(X=\frac{1}{2}, Y=1)=0.1$;
\par
б) $P(X=-1)=\sum\limits_{y\in\{0,1,2\}} P(X=-1,Y=y)=0.1+0.05+0=0.15$;
\par
в) $P(Y=1)=\sum\limits_{x\in\{-1,0.5,2\}} P(X=x,Y=1)=0.05+0.1+0.2=0.35$;
\par
г) $P(X^2+Y=2)=P(X=-1,Y=1)=0.05$. 
\section*{Задача 6}
{\bf Ответ: } верно.
\\
\\
{\bf Решение.}
\par
Заметим, что $f_{X,Y}(x,y)=\frac{1}{4}e^{-|x|-|y|}=\underbrace{\frac{1}{2}e^{-|x|}}_{f_X(x)}\cdot\underbrace{\frac{1}{2}e^{-|y|}}_{f_Y(y)}$. При этом $f_X(x) \geq 0\ \forall x\in\mathbb{R}$ и 
$$\int\limits_{-\infty}^{+\infty} f_X(x)\,dx=2\cdot\int\limits_0^{+\infty} \frac{1}{2}\cdot e^{-x}\,dx=-e^{-x}\bigg|_0^{+\infty}=0-(-1)=1$$
т.е. $f_X(x)$ - это плотность распределения величины $X$. Аналогично $f_Y(y)$ — это плотность распределения величины $Y$. Следовательно, $X$ и $Y$ независимы.
\section*{Задача 7}
{\bf Ответ: } $f_{X,Y}(x,y)=\frac{1}{\sqrt{2\pi}}e^{-\frac{x^2}{2}}\cdot\frac{1}{\pi(y^2+1)}$.
\\
\\
{\bf Решение.}
\par
Т.к. $X$ и $Y$ независимы, то плотность совместного распределения будет 
$$f_{X,Y}(x,y)=f_X(x)f_Y(y)=\frac{1}{\sqrt{2\pi}}e^{-\frac{x^2}{2}}\cdot\frac{1}{\pi(y^2+1)}.$$
\section*{Задача 8}
{\bf Ответ: } $\frac{1}{48}$.
\\
\\
{\bf Решение.}
\par
Заметим, что 
$$F_{X,Y}(x,y)=\frac{1}{\pi^2}\arctg(x) \arctg(y) + \frac{1}{2\pi}\arctg(x) + \frac{1}{2\pi}\arctg(y) + \frac{1}{4}=\underbrace{\left(\frac{1}{\pi}\arctg(x) + \frac{1}{2}\right)}_{F_X(x)}\cdot\underbrace{\left(\frac{1}{\pi}\arctg(y) + \frac{1}{2}\right)}_{F_Y(y)}$$
При этом функция $F_X(x)$ (как и функция $F_Y(y)$) имеет вид распределения Коши с параметрами $\gamma=1,\,x_0=0$, т.е. случайные величины $X,Y$ - независимы. Учитывая независимость $X$ и $Y$ совместная плотность будет произведением плотностей, т.е. имеет вид $f_{X,Y}(x,y)=f_X(x)f_Y(y)$, поэтому
$$P(1 < X \leq \sqrt{3},\,0<Y \leq 1)=\int\limits_{\substack{1 < x \leq \sqrt{3}\\0 < y \leq 1}} f_{X,Y}(x,y)\,dx dy=\int\limits_1^{\sqrt{3}} f_X(x)\,dx\int\limits_0^1 f_Y(y)\,dy=$$
$$=(F_X(\sqrt{3})-F_X(1))(F_Y(1)-F_Y(0))=\left(\frac{1}{\pi}\underbrace{\arctg\sqrt{3}}_{\frac{\pi}{3}}-\frac{1}{\pi}\underbrace{\arctg 1}_{\frac{\pi}{4}}\right)\cdot\left(\frac{1}{\pi}\underbrace{\arctg 1}_{\frac{\pi}{4}}-\frac{1}{\pi}\underbrace{\arctg 0}_{0}\right)=\left(\frac{1}{3}-\frac{1}{4}\right)\cdot\frac{1}{4}=\frac{1}{12}\cdot\frac{1}{4}=\frac{1}{48}.$$
\end{document}
