\documentclass{article}
\usepackage{graphicx} % Required for inserting images
\usepackage[T2A]{fontenc}
\usepackage[english, russian]{babel}

\setlength\parindent{1.5em}


\title{Домашнее задание 2}
\author{Андрей Зотов}
\date{Апрель 2023}

\begin{document}

\maketitle

\section*{Задача 1}
{\bf Ответ:} 20 опрошенных не знают ни английского, ни испанского.
\\
\\
{\bf Решение.} Пусть $C$ - множество всех опрошенных, $A$ - подмножество $C$ тех, кто знает английский, $B$ - подмножество $C$ тех, кто знает испанский. Тогда $|C|=250, |A|=210, |B|=100, |A\cap B|=80$ и искомое число опрошенных, не знающих ни английского, ни испанского, будет: $|C|-|A\cup B| = |C| - (|A| + |B| - |A\cap B|) = 250 - (210 + 100 - 80) = 20$.
\section*{Задача 2}
{\bf Ответ:} $151200$ способами можно заполнить вакансии.
\\
\\
{\bf Решение.} На первую вакансию можно взять любого из 10 кандидатов, на вторую любого из оставшихся 9 кандидатов, на третью любого из 8 и т.д. Т.е. всего имеется 10*9*8*7*6*5 = ($A_{10}^6$) = 151200 способов заполнить вакансии.
\section*{Задача 3}
{\bf Ответ:}  искомая вероятность будет 0.8488
\\
\\
{\bf Решение.} Элементарным исходом считаем случайный шестизначный код, поэтому пространство элементарных исходов $\Omega$ будет состоять из $10^6$ различных элементов (в каждый из 6 разрядов кода можно поместить любую из 10 цифр). Все элементарные исходы считаем равновероятными, поэтому вероятность любого элементарного исхода будет $\frac{1}{10^6}$, т.е. $\forall \omega \in \Omega\ P(\omega)=\frac{1}{10^6}$ и тогда, очевидно, $\sum_{\omega\in\Omega}P(\omega)=1$.
\par
Пусть событие $A = \{$В случайном шестизначном коде имеется хотя бы две одинаковые цифры$\}$, вероятность которого требуется найти. Тогда событие $B = \Omega\setminus A=\{$В случайном шестизначном коде все цифры разные$\}.$  По определению вероятность события $B$ будет $P(B)=\sum_{\omega\in B}P(\omega) = \frac{1}{10^6}*|B|$. 
\par
Найдем число элементарных исходов в событии $B$ (т.е. $|B|$): в первом разряде подходящего кода (элементарного исхода) может стоять любая из 10 цифр, во втором - любая из 9 цифр и т.д., т.е. число подходящих кодов будет $A_{10}^6=151200$ (см. Задачу 2).
\par
Таким образом $|B|=151200$  и следовательно $P(B)=\frac{151200}{10^6}=0.1512$. С другой стороны $P(B)=P(\Omega\setminus A)=1 - P(A)$, т.е. искомая вероятность будет $P(A)=1-0.1512=0.8488$.
\section*{Задача 4}
{\bf Ответ:}
\begin{itemize}
\item a) Среди первого миллиона натуральных чисел больше тех, в записи которых нет единицы.
\item б) Среди первых 10 миллионов натуральных чисел больше тех, в записи которых есть единица.
\end{itemize}
{\bf Решение. } a) Т.к. в нашем курсе считаем, что натуральные числа начинаются с нуля, то рассматриваемое множество (первый миллион натуральных чисел) будет $A=\{0, 1, 2, \dots, 999999\}$. При этом, если $B_0=\{x \in A|$ В записи $x$ нет единицы$\}$ и $B_1=\{x \in A|$В записи $x$ есть единица$\}$, то
$A=B_0 \cup B_1$ и т.к. $B_0 \cap B_1 = \emptyset$, то $|A| = |B_0| + |B_1| = 1000000$. Поэтому достаточно найти $|B_0|$.
\par
Заметим, что все множество $B_0$ можно разбить на 6 непересекающихся подмножеств $B_0^k=\{x \in B_0| x$ имеет $k$ знаков$\}$, тогда $|B_0|=\sum_{k=1}^6|B_0^k|$. При этом $B_0^1=\{0,2,3,4,5,6,7,8,9\}$, т.е. $|B_0^1|=9$. В $B_0^2$ входят двузначные числа, в старшем разряде которых не может быть 0 и 1, а в младшем не может быть 1, т.е. $|B_0^2|=8*9$ и аналогично $|B_0^3|=8*9^2$. Т.е., вообще говоря, если $k > 1$, то $|B_0^k|=8*9^{k-1}$.
\par
Таким образом $|B_0|=\sum_{k=1}^6|B_0^k|=9+8*\sum_{k=1}^5 9^k$, где последняя сумма - это сумма геометрической прогрессии, т.е. $|B_0|=9+8*\frac{9^6-9}{9-1}=9^6$ (что как бы говорит нам, что проще было рассматривать $B_0$ как множество всех шестизначных кодов, в записи которых не встречается единица - и их число, очевидно, $9^6$). И, следовательно, $|B_1|=|A|-|B_0|=10^6-9^6=468559$, что меньше, чем $|B_0|=9^6=531441$, т.е. среди первого миллиона натуральных чисел больше тех, в записи которых нет единицы.
\par
 б) В этом случае $A=\{0,1,2,\dots,9999999\}$ и $|A|=10^7$. По аналогии с подзадачей а) рассмотрим множества $B_0$ и $B_1$. И как было замечено в подзадаче а) $B_0$ - можно считать множеством семизначных кодов, в записи которых не встречается единица, т.е. $|B_0|=9^7=4782969$, что меньше чем $|B_1|=10^7-9^7=5217031$.
Таким образом среди первых 10 миллионов натуральных чисел больше тех, в записи которых есть единица.
\section*{Задача 5}
{\bf Ответ:} вероятность выпадения дубля при броске двух кубиков будет $\frac{1}{6}$
\\
\\
{\bf Решение.} Вероятностное пространство $\Omega$ задачи состоит из 36 элементарных исходов (число различных упорядоченных пар натуральных чисел от 1 до 6). Все элементарные исходы считаем равновероятными, поэтому $\forall \omega \in \Omega\ P(\omega)=1/36$ и тогда, очевидно, $\sum_{\omega\in\Omega}P(\omega)=1$. Рассмотрим событие $A=\{$При броске выпал дубль$\}$, вероятность которого и нужно найти, тогда $P(A)=\sum_{\omega \in A}P(\omega)=\frac{|A|}{36}$. При этом $A$ состоит из 6 элементарных исходов (6 возможных дублей $(1,1), (2,2),\dots, (6,6)$), т.е. $|A|=6$. Таким образом искомая вероятность $P(A)=6/36=1/6$.  
\section*{Задача 6}
{\bf Ответ:} а) вероятность $\frac{17}{27}\approx0.63$, б) вероятность $\frac{7}{27}\approx0.26$.
\\
\\
{\bf Решение.} Пространство элементарных исходов $\Omega$ участия команды в турнире состоит из упорядоченных четверок $(r_1,r_2,r_3,r_4)$, где $r_i$ - один из двух результатов $i$-го матча: либо выигрыш, либо поражение. Можем считать, что $r_i \in \{0,1\}$, где 0 - это поражение, а 1 - выигрыш. Тогда $|\Omega|$ - это число различных двоичных кодов длины 4, т.е. $2^4$, а вероятности элементарных исходов вычисляются с помощью дерева событий, например, $P(0101)=1/2*1/3*1/3*1/3=1/54$ (правило вычисления вероятности такое: первый множитель 1/2, далее переходы $0\rightarrow0$ и $1\rightarrow1$ дают множитель 2/3, а переходы $0\rightarrow1$ и $1\rightarrow0$ дают множитель 1/3). Т.к. на каждом шаге ветвления возникает множество исходов отдельного матча с суммарной вероятностью 1, то $\sum_{\omega\in\Omega}P(\omega)=1$, т.е. наша построенная система элементарных событий $\Omega$ с функцией $P$ будет конечным вероятностным пространством.
\begin{itemize}
\item a) Пусть событие $A = \{$Команда выиграла не менее двух игр$\}$, тогда событие $\bar A = \Omega \setminus A = \{$Команда выиграла не более одной игры за турнир$\}$, т.е. $\bar A=\{0000, 1000, 0100, 0010, 0001\}$. Поэтому $P(\bar A)=P(0000) + P(1000) + P(0100) + P(0010) + P(0001)=1/2*2/3*2/3*2/3 + 1/2*1/3*2/3*2/3 + 1/2*1/3*1/3*2/3+1/2*2/3*1/3*1/3+1/2*2/3*2/3*1/3=\frac{4}{27}+\frac{2}{27}+\frac{1}{27}+\frac{1}{27}+\frac{2}{27}=\frac{10}{27}$. А т.к. $P(\bar A)=1-P(A)$, то $P(A)=\frac{17}{27}$
\item б) Пусть событие $A=\{$Команда выиграла ровно две игры за турнир$\}$, тогда $A=\{1100, 1010, 1001, 0110, 0101, 0011\}$. Поэтому $P(A)=P(1100)+P(1010)+P(1001)+P(0110)+P(0101)+P(0011)=1/2*2/3*1/3*2/3+1/2*1/3*1/3*1/3+1/2*1/3*2/3*1/3+1/2*1/3*2/3*1/3+1/2*1/3*1/3*1/3+1/2*2/3*1/3*2/3=\frac{2}{27}+\frac{1}{54}+\frac{1}{27}+\frac{1}{27}+\frac{1}{54}+\frac{2}{27}$. Таким образом $P(A)=\frac{7}{27}$.
\end{itemize}
\section*{Задача 7}
{\bf Ответ:} a) $P(A)=\frac{7}{64}\approx 0.11$, б) $P(B)=\frac{219}{256}\approx0.86$.
\\
\\
{\bf Решение.} Пространство элементарных исходов $\Omega$ будем считать всевозможные восьмизначные двоичные коды, где 0 на $k$-м месте будет означать, что на $k$-м броске выпала решка, а 1 - что на $k$-м броске выпал орел. Тогда кол-во элементарных исходов будет $|\Omega|=2^8$. Все исходы считаем равновероятными, поэтому $\forall\omega\in\Omega\ P(\omega)=1/2^8$ и тогда, очевидно, $\sum_{\omega\in\Omega}P(\omega)=1$.
\begin{itemize}
\item а) Пусть событие $A=\{$Орел выпал 6 раз$\}$, тогда $P(A)=\sum_{\omega\in A}P(\omega)=\frac{|A|}{2^8}$, где $|A| = {8 \choose 6} = 28$ - кол-во разных восьмизначных двоичных кодов с ровно шестью единицами. Таким образом $P(A)=\frac{28}{2^8}=\frac{7}{64}\approx 0.109\dots$
\item б) Пусть событие $B=\{$Орел выпал не менее трех раз$\}$, тогда событие $\bar B=\Omega\setminus B=\{$Орел выпал ноль, один или два раза$\}$ и $P(\bar B)=\frac{|\bar B|}{2^8}$. Пусть событие $B_k=\{$Орел выпал $k$ раз$\}$, $8 \geq k \geq 0$, тогда $\bar B=B_0\cup B_1\cup B_2$ и т.к. $B_i\cap B_j = \emptyset$ при $i\neq j$, то $|\bar B| = |B_0|+|B_1|+|B_2|$. Учитывая, что $|B_k|={8 \choose k}$ ($8 \geq k \geq 0$), получаем $$P(\bar B)=\frac{{8 \choose 0} + {8 \choose 1} + {8 \choose 2}}{2^8}=\frac{1+8+28}{2^8}=\frac{37}{2^8}$$
С другой стороны $P(\bar B)=1 - P(B)$, т.е. $P(B)=1 - 37/256 = \frac{219}{256}\approx0.855\dots$
\end{itemize}

\end{document}
