\documentclass{article}
\usepackage{graphicx} % Required for inserting images

\usepackage[T2A]{fontenc}
\usepackage[utf8]{inputenc}
\usepackage[english, russian]{babel}
\usepackage{amsfonts}
\usepackage{amsmath}
\usepackage[left=2cm,right=2cm,
    top=2cm,bottom=2cm,bindingoffset=0cm]{geometry}
\setlength\parindent{1.5em}


\title{Домашнее задание 6 (линал)}
\author{Андрей Зотов}
\date{Июль 2023}

\begin{document}
\maketitle
\section*{Задача 1}
{\bf Ответ:} a) $A=\left(\begin{array}{rr}2 & 1\\1 & 2\end{array}\right)=CDC^T=\left(\begin{array}{rr}-1/\sqrt{2} & 1/\sqrt{2}\\1/\sqrt{2} & 1/\sqrt{2}\end{array}\right)\cdot\left(\begin{array}{rr}1 & 0\\0 & 3\end{array}\right)\cdot\left(\begin{array}{rr}-1/\sqrt{2} & 1/\sqrt{2}\\1/\sqrt{2} & 1/\sqrt{2}\end{array}\right)$;
\\
б) $A=\left(\begin{array}{rrr}0 & 0 & 1\\0 & 1 & 0\\1 & 0 & 0\end{array}\right)=CDC^T=\left(\begin{array}{rrr}-1/\sqrt{2} & 0 & 1/\sqrt{2}\\0 & 1 & 0\\1/\sqrt{2} & 0 & 1/\sqrt{2}\end{array}\right)\cdot\left(\begin{array}{rrr}-1 & 0 & 0\\0 & 1 & 0\\0 & 0 & 1\end{array}\right)\cdot\left(\begin{array}{rrr}-1/\sqrt{2} & 0 & 1/\sqrt{2}\\0 & 1 & 0\\1/\sqrt{2} & 0 & 1/\sqrt{2}\end{array}\right)$.
\\
\\
{\bf Решение.} a) Найдем собственные значения $A$:
$$\lambda E - A = \left(\begin{array}{cc}\lambda-2 & -1\\-1 & \lambda-2\end{array}\right)$$
$$\Downarrow$$
$$\chi_{A}(\lambda)=|\lambda E - A |=(\lambda-2)^2-1=(\lambda-1)(\lambda-3)=0$$
$$\Downarrow$$
$$\lambda\in\{1, 3\}$$
$$\Downarrow$$
$$D=\left(\begin{array}{rr}1 & 0\\0 & 3\end{array}\right)$$
\par
Имеем 2 разных собственных значения $\Rightarrow$ имеется два одномерных собственных подпространства $V_{\lambda}$ и т.к. матрица $A$ симметрична эти подпространства ортогональны. Найдем ортонормированный базис $\{e_1, e_2\}$ из собственных векторов:
\par
1) $V_{\lambda=1}=\{u\in\mathbb{R}^2|(A-E)u=0\}$
$$A-E=\left(\begin{array}{rr}1 & 1\\1 & 1\end{array}\right)\rightarrow\left(\begin{array}{rr}1 & 1\\0 & 0\end{array}\right)$$
$$\Downarrow$$
$$V_{\lambda=1}=\langle (-1, 1)^T \rangle$$
$$\Downarrow$$
$$e_1 = (-1/\sqrt{2}, 1/\sqrt{2})^T$$
\par
2) $V_{\lambda=3}=\{u\in\mathbb{R}^2|(A-3E)u=0\}$
$$A-3E=\left(\begin{array}{rr}-1 & 1\\1 & -1\end{array}\right)\rightarrow\left(\begin{array}{rr}1 & -1\\0 & 0\end{array}\right)$$
$$\Downarrow$$
$$V_{\lambda=3}=\langle (1, 1)^T \rangle$$
$$\Downarrow$$
$$e_2 = (1/\sqrt{2}, 1/\sqrt{2})^T$$
\par
Таким образом ортогональная матрица перехода будет $C=\left(\begin{array}{rr}-1/\sqrt{2} & 1/\sqrt{2}\\1/\sqrt{2} & 1/\sqrt{2}\end{array}\right)$ и поэтому:
$$A=\left(\begin{array}{rr}2 & 1\\1 & 2\end{array}\right)=CDC^T=\left(\begin{array}{rr}-1/\sqrt{2} & 1/\sqrt{2}\\1/\sqrt{2} & 1/\sqrt{2}\end{array}\right)\cdot\left(\begin{array}{rr}1 & 0\\0 & 3\end{array}\right)\cdot\left(\begin{array}{rr}-1/\sqrt{2} & 1/\sqrt{2}\\1/\sqrt{2} & 1/\sqrt{2}\end{array}\right).$$
\par
б) Найдем собственные значения $A$:
$$\lambda E - A = \left(\begin{array}{ccc}\lambda & 0 & -1\\0 & \lambda-1 & 0\\-1 & 0 & \lambda\end{array}\right)$$
$$\Downarrow$$
$$\chi_{A}(\lambda)=|\lambda E - A |=\lambda^2(\lambda-1)-(\lambda-1)=(\lambda-1)^2(\lambda+1)=0$$
$$\Downarrow$$
$$\lambda\in\{-1, 1\}$$
$$\Downarrow$$
$$D=\left(\begin{array}{rrr}-1 & 0 & 0\\0 & 1 & 0\\0 & 0 & 1\end{array}\right)$$
\par
Найдем ортонормированный базис $\{e_1, e_2, e_3\}$ из собственных векторов:
\par
1) $V_{\lambda=-1}=\{u\in\mathbb{R}^3|(A+E)u=0\}$
$$A+E=\left(\begin{array}{rrr}1 & 0 & 1\\0 & 2 & 0\\1 & 0 & 1\end{array}\right)\rightarrow\left(\begin{array}{rrr}1 & 0 & 1\\0 & 1 & 0\end{array}\right)$$
$$\Downarrow$$
$$V_{\lambda=-1}=\langle (-1, 0, 1)^T \rangle$$
$$\Downarrow$$
$$e_1 = (-1/\sqrt{2}, 0, 1/\sqrt{2})^T$$
\par
2) $V_{\lambda=1}=\{u\in\mathbb{R}^3|(A-E)u=0\}$
$$A-E=\left(\begin{array}{rrr}-1 & 0 & 1\\0 & 0 & 0\\1 & 0 & -1\end{array}\right)\rightarrow\left(\begin{array}{rrr}1 & 0 & -1\end{array}\right)$$
$$\Downarrow$$
$$V_{\lambda=1}=\langle (0, 1, 0)^T, (1, 0, 1)^T \rangle$$ 
$$\textrm{Как видно ФСР уже состоит из ортогональных векторов. Остается только их отнормировать.}$$
$$\Downarrow$$
$$e_2 = (0, 1, 0)^T,\ e_3 = (1/\sqrt{2}, 0, 1/\sqrt{2})^T$$
\par
Таким образом ортогональная матрица перехода будет $C=\left(\begin{array}{rrr}-1/\sqrt{2} & 0 & 1/\sqrt{2}\\0 & 1 & 0\\1/\sqrt{2} & 0 & 1/\sqrt{2}\end{array}\right)$ и поэтому:
$$A=\left(\begin{array}{rrr}0 & 0 & 1\\0 & 1 & 0\\1 & 0 & 0\end{array}\right)=CDC^T=\left(\begin{array}{rrr}-1/\sqrt{2} & 0 & 1/\sqrt{2}\\0 & 1 & 0\\1/\sqrt{2} & 0 & 1/\sqrt{2}\end{array}\right)\cdot\left(\begin{array}{rrr}-1 & 0 & 0\\0 & 1 & 0\\0 & 0 & 1\end{array}\right)\cdot\left(\begin{array}{rrr}-1/\sqrt{2} & 0 & 1/\sqrt{2}\\0 & 1 & 0\\1/\sqrt{2} & 0 & 1/\sqrt{2}\end{array}\right).$$

\section*{Задача 2}
{\bf Ответ:} a) $A=\left(\begin{array}{rrr}13 & 14 & 4\\14 & 24 & 18\\4 & 18 & 29\end{array}\right)=CDC^T=\left(\begin{array}{rrr}2/3 & 2/3 & 1/3\\-2/3 & 1/3 & 2/3\\1/3 & -2/3 & 2/3\end{array}\right)\cdot\left(\begin{array}{rrr}1 & 0 & 0\\0 & 16 & 0\\0 & 0 & 49\end{array}\right)\cdot\left(\begin{array}{rrr}2/3 & -2/3 & 1/3\\2/3 & 1/3 & -2/3\\1/3 & 2/3 & 2/3\end{array}\right)$;
\\
б) $B=\left(\begin{array}{rrr}3 & 2 & 0\\2 & 4 & 2\\0 & 2 & 5\end{array}\right)$.
\\
\\
{\bf Решение.} Найдем собственные значения $A$:
$$\lambda E - A = \left(\begin{array}{rrr}\lambda-13 & -14 & -4\\-14 & \lambda-24 & -18\\-4 & -18 & -29\end{array}\right)$$
$$\Downarrow$$
$$\chi_A(\lambda)=|\lambda E - A| = (\lambda-13)(\lambda-24)(\lambda-29)-14\cdot18\cdot4-14\cdot18\cdot4-16\cdot(\lambda-24)-18\cdot18\cdot(\lambda-13)-14\cdot14\cdot(\lambda-29)=0$$
$$\Updownarrow$$
$$(\lambda-13)(\lambda-24)(\lambda-29)-2016+384+4212+5684-16\lambda-324\lambda-196\lambda=0$$
$$\Updownarrow$$
$$(\lambda-13)(\lambda-24)(\lambda-29)+8264-536\lambda=0$$
$$\Updownarrow$$
$$\lambda^3+\lambda^2(-29-13-24)+\lambda(13\cdot24+13\cdot29+24\cdot29)-13\cdot24\cdot29+8264-536\lambda=0$$
$$\Updownarrow$$
$$\lambda^3-66\lambda^2+849\lambda-784=0$$
Заметим, что $\lambda=1$ является корнем характеристического многочлена поэтому, поделив этот многочлен на $\lambda-1$, получим квадратное уравнение:
$$\lambda^2-65\lambda+784=0$$
$$\Updownarrow$$
$$\lambda_{1,2}=\frac{65\pm\sqrt{1089}}{2}=\frac{65\pm33}{2}$$
$$\Updownarrow$$
$$\lambda\in\{16,49\}$$
Таким образом собственные значения $A$ будут $\lambda\in\{1,16,49\}\Rightarrow D=\left(\begin{array}{rrr}1 & 0 & 0\\0 & 16 & 0\\0 & 0 & 49\end{array}\right)$. Найдем ортонормированный базис $\{e_1,e_2,e_3\}$ из собственных векторов:
\par
1) $V_{\lambda=1}=\{u\in\mathbb{R}^3|(A-E)u=0\}$
$$A-E=\left(\begin{array}{rrr}12 & 14 & 4\\14 & 23 & 18\\4 & 18 & 28\end{array}\right)\rightarrow\left(\begin{array}{rrr}6 & 7 & 2\\14 & 23 & 18\\2 & 9 & 14\end{array}\right)\rightarrow\left(\begin{array}{rrr}6 & 7 & 2\\2 & 9 & 14\\2 & 9 & 14\end{array}\right)\rightarrow\left(\begin{array}{rrr}6 & 7 & 2\\2 & 9 & 14\end{array}\right)\rightarrow\left(\begin{array}{rrr}4 & -2 & -12\\2 & 9 & 14\end{array}\right)\rightarrow$$
$$\left(\begin{array}{rrr}2 & -1 & -6\\2 & 9 & 14\end{array}\right)\rightarrow\left(\begin{array}{rrr}2 & -1 & -6\\0 & 10 & 20\end{array}\right)\rightarrow\left(\begin{array}{rrr}1 & -0.5 & -3\\0 & 1 & 2\end{array}\right)$$
$$\Downarrow$$
$$\textrm{ФСР:}\ x_3=1\Rightarrow x_2 = -2,\ x_1=2.$$
$$\Downarrow$$
$$V_{\lambda=1}=\langle (2,-2,1)^T \rangle$$
$$\Downarrow$$
$$e_1=(2/3,-2/3,1/3)^T$$
\par
2) $V_{\lambda=16}=\{u\in\mathbb{R}^3|(A-16E)u=0\}$
$$A-16E=\left(\begin{array}{rrr}-3 & 14 & 4\\14 & 8 & 18\\4 & 18 & 13\end{array}\right)\rightarrow\left(\begin{array}{rrr}-3 & 14 & 4\\14 & 8 & 18\\1 & 32 & 17\end{array}\right)\rightarrow\left(\begin{array}{rrr}1 & 32 & 17\\14 & 8 & 18\\0 & 110 & 55\end{array}\right)\rightarrow\left(\begin{array}{rrr}1 & 32 & 17\\7 & 4 & 9\\0 & 2 & 1\end{array}\right)\rightarrow\left(\begin{array}{rrr}1 & 32 & 17\\0 & -220 & -110\\0 & 2 & 1\end{array}\right)\rightarrow$$
$$\left(\begin{array}{rrr}1 & 32 & 17\\0 & 1 & 0.5\end{array}\right)$$
$$\Downarrow$$
$$\textrm{ФСР:}\ x_3=1\Rightarrow x_2 = -0.5,\ x_1=-1$$
$$\Downarrow$$
$$V_{\lambda=16}=\langle (-1,-0.5, 1)^T \rangle$$
$$\Downarrow$$
$$e_2=(2/3,1/3,-2/3)^T$$
\par
3) $V_{\lambda=49}=\{u\in\mathbb{R}^3|(A-49E)u=0\}$. Нам уже известно, что $V_{\lambda=49}$ одномерно (т.к. у нас 3 разных собственных значения у оператора действующего в $\mathbb{R}^3$) и ортогонально по отношению к $V_{\lambda=1}$ и $V_{\lambda=16}$ (это вытекает из симметричности $A$), поэтому базисный собственный вектор для $V_{\lambda=49}$ и вектор $e_3$ проще найти из двух линейных уравнений $(e_3, e_2)=0$ и $(e_3, e_1)=0$, которым соответствует матрица:
$$\left(\begin{array}{rrr}2 & -2 & 1\\2 & 1 & -2\end{array}\right)\rightarrow\left(\begin{array}{rrr}2 & -2 & 1\\0 & 3 & -3\end{array}\right)\rightarrow\left(\begin{array}{rrr}1 & -1 & 0.5\\0 & 1 & -1\end{array}\right)$$
$$\Downarrow$$
$$\textrm{ФСР: } x_3=1\Rightarrow x_2=1,\ x_1=0.5$$
$$\Downarrow$$
$$V_{\lambda=49}=\langle (0.5, 1, 1)^T \rangle=\langle (1, 2, 2)^T \rangle$$
$$\Downarrow$$
$$e_3=(1/3, 2/3, 2/3)$$
\par
Таким образом ортогональная матрица перехода будет $C=\left(\begin{array}{rrr}2/3 & 2/3 & 1/3\\-2/3 & 1/3 & 2/3\\1/3 & -2/3 & 2/3\end{array}\right)$ и поэтому:
$$A=\left(\begin{array}{rrr}13 & 14 & 4\\14 & 24 & 18\\4 & 18 & 29\end{array}\right)=CDC^T=\left(\begin{array}{rrr}2/3 & 2/3 & 1/3\\-2/3 & 1/3 & 2/3\\1/3 & -2/3 & 2/3\end{array}\right)\cdot\left(\begin{array}{rrr}1 & 0 & 0\\0 & 16 & 0\\0 & 0 & 49\end{array}\right)\cdot\left(\begin{array}{rrr}2/3 & -2/3 & 1/3\\2/3 & 1/3 & -2/3\\1/3 & 2/3 & 2/3\end{array}\right).$$
\par
б) Пусть $D'$ диагональная матрица, на диагонали которой стоят квадратные корни из собственных значений $A$ (вообще говоря с любым знаком) в том же порядке, как и в матрице $D$, тогда, очевидно, $D'^2=D$. Кроме того, $C$ - ортогональная матрица, т.е. $C^TC=E$. Поэтому, если $B=CD'C^T$, то $B^2=CD'C^TCD'C^T=CD'^2C^T=CDC^T=A$.  Таким образом, одним из возможных значений $B$ будет:
$$B=CD'C^T=\left(\begin{array}{rrr}2/3 & 2/3 & 1/3\\-2/3 & 1/3 & 2/3\\1/3 & -2/3 & 2/3\end{array}\right)\cdot\left(\begin{array}{rrr}1 & 0 & 0\\0 & 4 & 0\\0 & 0 & 7\end{array}\right)\cdot\left(\begin{array}{rrr}2/3 & -2/3 & 1/3\\2/3 & 1/3 & -2/3\\1/3 & 2/3 & 2/3\end{array}\right)=\left(\begin{array}{rrr}3 & 2 & 0\\2 & 4 & 2\\0 & 2 & 5\end{array}\right).$$
\section*{Задача 3}
{\bf Ответ:}
\begin{itemize}
    \item  a) $A=\left(\begin{array}{rrr}5 & 1 & 8\\7 & 5 & 4\end{array}\right)$
    \begin{itemize}
        \item Сингулярное разложение $A=\left(\begin{array}{rr}1/\sqrt{2} & -1/\sqrt{2}\\1/\sqrt{2} & 1/\sqrt{2}\end{array}\right)\cdot\left(\begin{array}{rrr}\sqrt{162} & 0 & 0\\0 & \sqrt{18} & 0\end{array}\right)\cdot\left(\begin{array}{rrr}2/3 & 1/3 & 2/3\\1/3 & 2/3 & -2/3\\-2/3 & 2/3 & 1/3\end{array}\right)$.
        \item Усеченное сингулярное разложение $A=\left(\begin{array}{rr}1/\sqrt{2} & -1/\sqrt{2}\\1/\sqrt{2} & 1/\sqrt{2}\end{array}\right)\cdot\left(\begin{array}{rr}\sqrt{162} & 0\\0 & \sqrt{18}\end{array}\right)\cdot\left(\begin{array}{rrr}2/3 & 1/3 & 2/3\\1/3 & 2/3 & -2/3\end{array}\right)$.
    \end{itemize}
    \item б) $A=\left(\begin{array}{rrrr}1 & 1 & 2 & 2\\2 & 2 & 1 & 1\end{array}\right)$
    \begin{itemize}
        \item Сингулярное разложение $A=\left(\begin{array}{rr}1/\sqrt{2} & -1/\sqrt{2}\\1/\sqrt{2} & 1/\sqrt{2}\end{array}\right)\cdot\left(\begin{array}{rrrr}\sqrt{18} & 0 & 0 & 0\\0 & \sqrt{2} & 0 & 0\end{array}\right)\cdot\left(\begin{array}{rrrr}0.5 & 0.5 & 0.5 & 0.5\\0.5 & 0.5 & -0.5 & -0.5\\-0.5 & 0.5 & -0.5 & 0.5\\-0.5 & 0.5 & 0.5 & -0.5\end{array}\right)$.
        \item Усеченное сингулярное разложение $A=\left(\begin{array}{rr}1/\sqrt{2} & -1/\sqrt{2}\\1/\sqrt{2} & 1/\sqrt{2}\end{array}\right)\cdot\left(\begin{array}{rrrr}\sqrt{18} & 0 &\\0 & \sqrt{2}\end{array}\right)\cdot\left(\begin{array}{rrrr}0.5 & 0.5 & 0.5 & 0.5\\0.5 & 0.5 & -0.5 & -0.5\end{array}\right)$.
    \end{itemize}
\end{itemize}

{\bf Решение.} a) Пусть $A=\left(\begin{array}{rrr}5 & 1 & 8\\7 & 5 & 4\end{array}\right)$. Ищем сингулярное разложение $A=C\Lambda D^T$, где $C$ ортогональная матрица $2\times2$, $\Lambda$ - матрица $2\times3$ с сингулярными числами на главной диагонали (остальные значения 0) и $D$ - ортогональная матрица $3\times3$.
\par
Найдем сначала собственные значения $S=AA^T$:
$$S=AA^T=\left(\begin{array}{rr}90 & 72\\72 & 90\end{array}\right)\Rightarrow \chi_S(\lambda)=|\lambda E - S|=(\lambda-90)^2-72^2=(\lambda-162)(\lambda-18)=0$$
$$\Downarrow$$
$$\lambda \in \{162, 18\}$$
$$\Downarrow$$
$$\Lambda=\left(\begin{array}{rrr}\sqrt{162} & 0 & 0\\0 & \sqrt{18} & 0\end{array}\right)$$
\par
Найдем ортогональную матрицу $C$ (ортонормированный базис из собственных векторов $\{e_1, e_2\}$, в котором $S$ имеет диагональный вид):
\par
1) $V_{\lambda=18}=\{u\in\mathbb{R}^2|(S-18E)u=0\}$
$$S-18E=\left(\begin{array}{rr}72 & 72\\72 & 72\end{array}\right)\rightarrow \left(\begin{array}{rr}1 & 1\end{array}\right)$$
$$\Downarrow$$
$$V_{\lambda=18}=\langle (-1, 1)^T \rangle$$
$$\Downarrow$$
$$e_2 = (-1/\sqrt{2}, 1/\sqrt{2})\  (\textrm{индекс 2, т.к. 18 второе по величине собственное значение})$$
\par
2) $V_{\lambda=162}=\{u\in\mathbb{R}^2|(S-162E)u=0\}$. Т.к. $S$ симметрична, то $V_{\lambda=162}\bot V_{\lambda=18}=\langle (-1, 1)^T \rangle$ поэтому, очевидно, что $V_{\lambda=162}=\langle (1,1)^T \rangle$. Значит $e_1=(1/\sqrt{2}, 1/\sqrt{2})^T$.
\par
Таким образом
$$C=\left(\begin{array}{rr}1/\sqrt{2} & -1/\sqrt{2}\\1/\sqrt{2} & 1/\sqrt{2}\end{array}\right)$$
\par
Найдем теперь $D^T$. Если $D_1^T$ - это первая строка $D^T$, то $D_1^T$ в точности равна первой строке матрицы $C^TA=\frac{1}{\sqrt{2}}\left(\begin{array}{rrr}12 & 6 & 12\\2 & 4 & -4\end{array}\right)$, деленной на $\sqrt{162}$ (первое по величине сингулярное число), т.е. 
$$D_1^T=\frac{1}{\sqrt{162}\cdot\sqrt{2}}(12, 6, 12)=(2/3,1/3,2/3)$$
\par
Аналогично находим $D_2^T$ (вторая строка $D^T$):
$$D_2^T=\frac{1}{\sqrt{18}\cdot\sqrt{2}}(2, 4, -4)=(1/3,2/3,-2/3)$$
\par
Осталось достроить $D_3^T$. Т.к. $D_3^T\bot D_2^T$ и $D_3^T\bot D_1^T$, то координаты $D_3^T$ являются решением ОСЛУ, которой соответствует матрица:
$$\left(\begin{array}{rrr}2 & 1 & 2\\1 & 2 & -2\end{array}\right)\rightarrow\left(\begin{array}{rrr}1 & 2 & -2\\2 & 1 & 2\end{array}\right)\rightarrow\left(\begin{array}{rrr}1 & 2 & -2\\0 & -3 & 6\end{array}\right)\rightarrow\left(\begin{array}{rrr}1 & 2 & -2\\0 & -1 & 2\end{array}\right)$$
$$\Downarrow$$
$$\textrm{ФСР: } (-2,2,1)^T$$
$$\Downarrow$$
$$D_3^T=(-2/3, 2/3, 1/3)$$
\par
Таким образом
$$D^T=\left(\begin{array}{rrr}2/3 & 1/3 & 2/3\\1/3 & 2/3 & -2/3\\-2/3 & 2/3 & 1/3\end{array}\right).$$
\par
Поэтому сингулярное разложение будет:
$$A=\left(\begin{array}{rr}1/\sqrt{2} & -1/\sqrt{2}\\1/\sqrt{2} & 1/\sqrt{2}\end{array}\right)\cdot\left(\begin{array}{rrr}\sqrt{162} & 0 & 0\\0 & \sqrt{18} & 0\end{array}\right)\cdot\left(\begin{array}{rrr}2/3 & 1/3 & 2/3\\1/3 & 2/3 & -2/3\\-2/3 & 2/3 & 1/3\end{array}\right).$$
\par
А вытекающее из него усеченное сингулярное разложение будет:
$$A=\left(\begin{array}{rr}1/\sqrt{2} & -1/\sqrt{2}\\1/\sqrt{2} & 1/\sqrt{2}\end{array}\right)\cdot\left(\begin{array}{rr}\sqrt{162} & 0\\0 & \sqrt{18}\end{array}\right)\cdot\left(\begin{array}{rrr}2/3 & 1/3 & 2/3\\1/3 & 2/3 & -2/3\end{array}\right).$$
\par
б) Действуем так же, как в задаче a) с аналогичными обозначениями. Для краткости опущу пояснения, т.к. по сути они такие же как и в задаче a).
$$A=\left(\begin{array}{rrrr}1 & 1 & 2 & 2\\2 & 2 & 1 & 1\end{array}\right)$$
$$\Downarrow$$
$$S=AA^T=\left(\begin{array}{rr}10 & 8\\8 & 10\end{array}\right)$$
$$\Downarrow$$
$$\chi_S(\lambda)=(\lambda-10)^2-8^2=(\lambda-18)(\lambda-2)$$
$$\Downarrow$$
$$\lambda\in\{18,2\}$$
$$\Downarrow$$
$$\Lambda=\left(\begin{array}{rrrr}\sqrt{18} & 0 & 0 & 0\\0 & \sqrt{2} & 0 & 0\end{array}\right)$$
\par
1) $V_{\lambda=18}=\{u\in\mathbb{R}^2|(S-18E)u=0\}$
$$S-18E=\left(\begin{array}{rr}-8 & 8\\8 & -8\end{array}\right)\rightarrow\left(\begin{array}{rr}1 & -1\end{array}\right)$$
$$\Downarrow$$
$$V_{\lambda=18}=\langle (1,1)^T \rangle$$
$$\Downarrow$$
$$e_1=(1/\sqrt{2}, 1/\sqrt{2})$$
\par
2)  $V_{\lambda=2}=\{u\in\mathbb{R}^2|(S-2E)u=0\}$
$$V_{\lambda=2}\bot V_{\lambda=18}$$
$$\Downarrow$$
$$V_{\lambda=2}=\langle (-1,1)^T \rangle$$
$$\Downarrow$$
$$e_2=(-1/\sqrt{2}, 1/\sqrt{2})$$
\par
Таким образом
$$C=\left(\begin{array}{rr}1/\sqrt{2} & -1/\sqrt{2}\\1/\sqrt{2} & 1/\sqrt{2}\end{array}\right)$$
$$\Downarrow$$
$$C^TA=\frac{1}{\sqrt{2}}\left(\begin{array}{rrrr}3 & 3 & 3 & 3\\1 & 1 & -1 & -1\end{array}\right)$$
$$\Downarrow$$
$$D_1^T=\frac{1}{\sqrt{18}\cdot\sqrt{2}}(3,3,3,3)=(0.5,0.5,0.5,0.5)$$
$$D_2^T=\frac{1}{\sqrt{2}\cdot\sqrt{2}}(1,1,-1,-1)=(0.5,0.5,-0.5,-0.5)$$
$$D_3^T,D_4^T\bot\langle D_1^T, D_2^T \rangle$$
$$\Downarrow$$
$$\textrm{Компоненты } D_3^T, D_4^T \textrm{ удовлетворяют ОСЛУ с матрицей:}$$
$$\left(\begin{array}{rrrr}1 & 1 & 1 & 1\\1 & 1 & -1 & -1\end{array}\right)\rightarrow\left(\begin{array}{rrrr}1 & 1 & 1 & 1\\0 & 0 & 1 & 1\end{array}\right)$$
$$\Downarrow$$
$$\textrm{ФСР: } u_1=(-1,1,0,0)^T, u_2=(0, 0, -1, 1)^T, \textrm{при этом видно, что } u_1\bot u_2.$$
$$\Downarrow$$
$$\textrm{Чтобы получить более красивый вид $D$ перейдем к другому базису пространства решений:}$$ 
$$v_1=u_1+u_2=(-1,1,-1,1)^T,\ v_2=u_1-u_2=(-1, 1, 1, -1)^T, \textrm{при этом видно, что } v_1\bot v_2.$$
$$\Downarrow$$
$$D_3^T=(-0.5,0.5,-0.5,0.5)$$
$$D_4^T=(-0.5,0.5,0.5,-0.5)$$
$$\Downarrow$$
$$D^T=\left(\begin{array}{rrrr}0.5 & 0.5 & 0.5 & 0.5\\0.5 & 0.5 & -0.5 & -0.5\\-0.5 & 0.5 & -0.5 & 0.5\\-0.5 & 0.5 & 0.5 & -0.5\end{array}\right)$$
\par
Поэтому сингулярное разложение будет:
$$A=\left(\begin{array}{rr}1/\sqrt{2} & -1/\sqrt{2}\\1/\sqrt{2} & 1/\sqrt{2}\end{array}\right)\cdot\left(\begin{array}{rrrr}\sqrt{18} & 0 & 0 & 0\\0 & \sqrt{2} & 0 & 0\end{array}\right)\cdot\left(\begin{array}{rrrr}0.5 & 0.5 & 0.5 & 0.5\\0.5 & 0.5 & -0.5 & -0.5\\-0.5 & 0.5 & -0.5 & 0.5\\-0.5 & 0.5 & 0.5 & -0.5\end{array}\right).$$
\par
А вытекающее из него усеченное сингулярное разложение будет:
$$A=\left(\begin{array}{rr}1/\sqrt{2} & -1/\sqrt{2}\\1/\sqrt{2} & 1/\sqrt{2}\end{array}\right)\cdot\left(\begin{array}{rrrr}\sqrt{18} & 0 &\\0 & \sqrt{2}\end{array}\right)\cdot\left(\begin{array}{rrrr}0.5 & 0.5 & 0.5 & 0.5\\0.5 & 0.5 & -0.5 & -0.5\end{array}\right).$$
\section*{Задача 4}
{\bf Ответ:} $A_1=\left(\begin{array}{rrr}6 & 3 & 6\\6 & 3 & 6\end{array}\right)$.
\\
\\
{\bf Решение.} Усеченное сингулярное разложение для $A$, найденное в задаче 3а) будет:
$$A=\left(\begin{array}{rr}1/\sqrt{2} & -1/\sqrt{2}\\1/\sqrt{2} & 1/\sqrt{2}\end{array}\right)\cdot\left(\begin{array}{rr}\sqrt{162} & 0\\0 & \sqrt{18}\end{array}\right)\cdot\left(\begin{array}{rrr}2/3 & 1/3 & 2/3\\1/3 & 2/3 & -2/3\end{array}\right).$$
Поэтому (оставляя ненулевым только первое самое большое сингулярное число) получаем:
$$A_1 = \left(\begin{array}{r}1\\1\end{array}\right)\cdot\frac{\sqrt{162}}{\sqrt{2}}\cdot\left(\begin{array}{rrr}2/3 & 1/3 & 2/3\end{array}\right)=\left(\begin{array}{rrr}6 & 3 & 6\\6 & 3 & 6\end{array}\right).$$
\section*{Задача 5}
{\bf Ответ:} $x_1^2+x_2^2-x_3^2$.
\\
\\
{\bf Решение.} Исходной квадратичной форме соответствует симметрическая билинейная форма $b(x,y)=x^TBy=x^T\left(\begin{array}{rrr}1 & 1 & 2\\1 & 2 & 1\\2 & 1 & 1\end{array}\right)y$, где $x,y\in\mathbb{R}^3$. Приведем матрицу $B$ к диагональному нормальному виду симметрическим методом Гаусса:
$$B=\left(\begin{array}{rrr}1 & 1 & 2\\1 & 2 & 1\\2 & 1 & 1\end{array}\right)\rightarrow\left(\begin{array}{rrr}1 & 1 & 2\\0 & 1 & -1\\2 & 1 & 1\end{array}\right)\rightarrow\left(\begin{array}{rrr}1 & 0 & 2\\0 & 1 & -1\\2 & -1 & 1\end{array}\right)\rightarrow\left(\begin{array}{rrr}1 & 0 & 2\\0 & 1 & -1\\0 & -1 & -1\end{array}\right)\rightarrow\left(\begin{array}{rrr}1 & 0 & 0\\0 & 1 & -1\\0 & -1 & -1\end{array}\right)\rightarrow$$
$$\left(\begin{array}{rrr}1 & 0 & 0\\0 & 1 & -1\\0 & 0 & -2\end{array}\right)\rightarrow\left(\begin{array}{rrr}1 & 0 & 0\\0 & 1 & 0\\0 & 0 & -2\end{array}\right)\rightarrow\left(\begin{array}{rrr}1 & 0 & 0\\0 & 1 & 0\\0 & 0 & -1\end{array}\right)$$
\par
Таким образом нормальный вид исходной квадратичной формы будет $x_1^2+x_2^2-x_3^2$.
\par
Метод Якоби позволяет проверить ответ:
$$\Delta_1=1,\ \Delta_2=2-1=1,\ \Delta_3=2+2+2-8-1-1=-2$$
$$\Downarrow$$
$$\lambda_1=\Delta_1=1,\ \lambda_2=\frac{\Delta_2}{\Delta_1}=1,\ \lambda_3=\frac{\Delta_3}{\Delta_2}=-2$$
$$\Downarrow$$
$$\textrm{нормальный вид формы: }x_1^2+x_2^2-x_3^2$$

\end{document}
