\documentclass{article}
\usepackage{graphicx} % Required for inserting images

\usepackage[T2A]{fontenc}
\usepackage[utf8]{inputenc}
\usepackage[english, russian]{babel}
\usepackage{amsfonts}
\usepackage{amsmath}
\usepackage[left=2cm,right=2cm,
    top=2cm,bottom=2cm,bindingoffset=0cm]{geometry}
\setlength\parindent{1.5em}
\DeclareMathOperator{\sgn}{sgn}
\DeclareMathOperator{\Spec}{Spec}

\title{Домашнее задание 2 (линал)}
\author{Андрей Зотов}
\date{Июнь 2023}

\begin{document}

\maketitle

\section*{Задача 1}
{\bf Ответ:} -8.
\\
\\
{\bf Решение.} $\det \left(\begin{array}{rrr}1 & 2 & 3\\5 & 1 & 4\\3 & 2 & 5\end{array}\right)=5 + 24 + 30 - 9 - 8 - 50 = 59 - 67 = -8$.
\section*{Задача 2}
{\bf Ответ:} $\chi_A(\lambda) = (\lambda-1)(\lambda-2)(\lambda-3) = \lambda^3-6\lambda^2+11\lambda-6;\ \Spec A=\{1, 2, 3\}$.
\\
\\
{\bf Решение.} Пусть $A =\left(\begin{array}{rrr}1 & 2 & -3\\-5 & 1 & -4\\0 & -2 & 4\end{array}\right)$, тогда  характеристический многочлен матрицы $A$ по определению будет: $\chi_A(\lambda)=\det(\lambda E - A) = \det\left(\begin{array}{ccc}\lambda-1 & -2 & 3\\5 & \lambda-1 & 4\\0 & 2 & \lambda-4\end{array}\right)=(\lambda-1)^2(\lambda-4)+30-8(\lambda-1)+10(\lambda-4)=(\lambda-1)^2(\lambda-4)+30-8(\lambda-1)+10(\lambda-1)-30 = (\lambda-1)((\lambda-1)(\lambda-4)+2)=(\lambda-1)(\lambda^2-5\lambda+6)$.
\\
Вторая скобка - это квадратный трехчлен, его корни будут:
$$\lambda_{1,2}=\frac{5\pm\sqrt{25-4*6}}{2}=\frac{5\pm 1}{2} \Leftrightarrow \lambda \in \{2, 3\}$$
\\
Таким образом
$$\chi_A(\lambda) = (\lambda-1)(\lambda-2)(\lambda-3) = \lambda^3-6\lambda^2+11\lambda-6$$
Т.е. корни $\chi_A(\lambda)$ будут $\lambda \in \{1, 2, 3\}$, поэтому $\Spec A = \{1, 2, 3\}$.
\\
\section*{Задача 3}
{\bf Ответ:} $2a-8b+c+5d$.
\\
\\
{\bf Решение.} Разложение по 2-му столбцу будет иметь вид:
$$\det\left(\begin{array}{rrrr}5 & a & 2 & -1\\4 & b & 4 & -3\\2 & c & 3 & -2\\4 & d & 5 & -4 \end{array}\right)=a(-1)^{1+2}\left|\begin{array}{rrr}4 & 4 & -3\\2 & 3 & -2\\4 & 5 & -4\end{array}\right|+b(-1)^{2+2}\left|\begin{array}{rrr}5 & 2 & -1\\2 & 3 & -2\\4 & 5 & -4\end{array}\right|+c(-1)^{3+2}\left|\begin{array}{rrr}5 & 2 & -1\\4 & 4 & -3\\4 & 5 & -4\end{array}\right|+d(-1)^{4+2}\left|\begin{array}{rrr}5 & 2 & -1\\4 & 4 & -3\\2 & 3 & 
-2\end{array}\right|$$
\par
Найдем все определители в этом выражении:
$$\left|\begin{array}{rrr}4 & 4 & -3\\2 & 3 & -2\\4 & 5 & -4\end{array}\right|=\left|\begin{array}{rrr}4 & 4 & -3\\2 & 3 & -2\\0 & -1 & 0\end{array}\right|=6-8=-2$$
$$\left|\begin{array}{rrr}5 & 2 & -1\\2 & 3 & -2\\4 & 5 & -4\end{array}\right|=\left|\begin{array}{rrr}5 & 2 & -1\\2 & 3 & -2\\0 & -1 & 0\end{array}\right|=2-10=-8$$
$$\left|\begin{array}{rrr}5 & 2 & -1\\4 & 4 & -3\\4 & 5 & -4\end{array}\right|=\left|\begin{array}{rrr}5 & 2 & -1\\4 & 4 & -3\\0 & 1 & -1\end{array}\right|=\left|\begin{array}{rrr}5 & 1 & 0\\4 & 4 & -3\\0 & 1 & -1\end{array}\right|=-20+15+4=-1$$
$$\left|\begin{array}{rrr}5 & 2 & -1\\4 & 4 & -3\\2 & 3 & 
-2\end{array}\right|=\left|\begin{array}{rrr}5 & 2 & -1\\0 & -2 & 1\\2 & 3 & 
-2\end{array}\right|=\left|\begin{array}{rrr}5 & 2 & -1\\0 & -2 & 1\\2 & -1 & 
0\end{array}\right|=4-4+5=5$$
\par
Таким образом
$$\det\left(\begin{array}{rrrr}5 & a & 2 & -1\\4 & b & 4 & -3\\2 & c & 3 & -2\\4 & d & 5 & -4 \end{array}\right)=2a-8b+c+5d.$$
\\
\section*{Задача 4}
{\bf Ответ:} $-t^5+a_1a_2a_3a_4a_5$.
\\
\\
{\bf Решение.} Пусть $X=\left(\begin{array}{rrrrr}-t & 0 & 0 & 0 & a_1\\a_2 & -t & 0 & 0 & 0\\0 & a_3 & -t & 0 & 0\\0 & 0 & a_4 & -t & 0\\0 & 0 & 0 & a_5 & -t\end{array}\right)$. Вычислим определитель, разлагая матрицу по первой строке:
$$\det X = \left|\begin{array}{rrrrr}-t & 0 & 0 & 0 & a_1\\a_2 & -t & 0 & 0 & 0\\0 & a_3 & -t & 0 & 0\\0 & 0 & a_4 & -t & 0\\0 & 0 & 0 & a_5 & -t\end{array}\right|=-t\cdot(-1)^{1+1}\left|\begin{array}{rrrr}-t & 0 & 0 & 0\\a_3 & -t & 0 & 0\\0 & a_4 & -t & 0\\0 & 0 & a_5 & -t\end{array}\right| + a_1\cdot(-1)^{1+5}\left|\begin{array}{rrrr}a_2 & -t & 0 & 0\\0 & a_3 & -t & 0\\0 & 0 & a_4 & -t\\0 & 0 & 0 & a_5\end{array}\right|$$
Обе матрицы в правой части имеют треугольный вид, поэтому их определители - это произведения диагональных элементов. Таким образом:
$$\det X = -t \cdot t^4 + a_1 \cdot a_2a_3a_4a_5=-t^5+a_1a_2a_3a_4a_5$$
\section*{Задача 5}
{\bf Ответ:} 0.
\\
\\
{\bf Решение.} Рассмотрим матрицу $A$ размера $n\times n$ , где  $n > 1$.  Пусть $A_i$ - i-я строка матрицы $A$. Если к 1-й строке прибавить любую другую строку, умноженную на коэффициент, то определитель не меняется, поэтому:
$$\det A = \det\left(\begin{array}{c}A_1 \\ \vdots \\ A_n\end{array}\right)=\det\left(\begin{array}{c}A_1 - A_2 + A_3 - \ldots - (-1)^n A_n\\ \vdots \\ A_n\end{array}\right)$$
Но строка $A'_1 = A_1 - A_2 + A_3 - \ldots - (-1)^n A_n$ - это сумма всех нечетных строк минус сумма всех четных строк матрицы $A$, и т.к. эти две суммы по условию равны как вектора, то $A'_1$ - это нулевой вектор (или строка) длины $n$, т.е. $A'_1 = \underbrace{(0, 0, \dots, 0)}_{n}$. А т.к. определитель матрицы с нулевой строкой равен 0, то и $\det A = 0$.
\\
\section*{Задача 6}
{\bf Ответ:} $A^{-1}=\left(\begin{array}{ccc}1 & -1 & 0\\ 0 & 1 & 0\\ 0 & -1 & 1/3\end{array}\right)$.
\\
\\
{\bf Решение.} a) $(A|E)=\left(\begin{array}{ccc|ccc}1 & 1 & 0 & 1 & 0  & 0\\ 0 & 1 & 0 & 0 & 1 & 0\\0 & 3 & 3 & 0 & 0 & 1\end{array}\right)\Leftrightarrow\left(\begin{array}{ccc|ccc}1 & 1 & 0 & 1 & 0 & 0\\0 & 1 & 0 & 0 & 1 & 0\\0 & 0 & 3 & 0 & -3 & 1\end{array}\right)\Leftrightarrow\left(\begin{array}{ccc|ccc}1 & 1 & 0 & 1 & 0 & 0\\0 & 1 & 0 & 0 & 1 & 0\\0 & 0 & 1 & 0 & -1 & 1/3\end{array}\right)\Leftrightarrow\left(\begin{array}{ccc|ccc}1 & 0 & 0 & 1 & -1  & 0\\0 & 1 & 0 & 0 & 1 & 0\\0 & 0 & 1 & 0 & -1 & 1/3\end{array}\right)=(E|A^{-1}) \Rightarrow A^{-1}=\left(\begin{array}{ccc}1 & -1 & 0\\ 0 & 1 & 0\\ 0 & -1 & 1/3\end{array}\right)$.
\par
б) $\det A = \left|\begin{array}{rrr}1 & 1 & 0\\0 & 1 & 0\\0 & 3 & 3\end{array}\right|=1\cdot\left|\begin{array}{rr}1 & 0\\3 & 3\end{array}\right|=3$. Поэтому, если $\tilde A = \textrm{матрица из алгебраических дополнений } =\left(\begin{array}{ccc}A_{11} & A_{12} & A_{13}\\A_{21} & A_{22} & A_{23}\\A_{31} & A_{32} & A_{33}\end{array}\right)$, где
$$A_{11}=(-1)^{1+1}\left|\begin{array}{ccc|ccc}1&0\\3&3\end{array}\right|=3,\ A_{12}=(-1)^{1+2}\left|\begin{array}{ccc|ccc}0&0\\0&3\end{array}\right|=0,\ A_{13}=(-1)^{1+3}\left|\begin{array}{ccc|ccc}0&1\\0&3\end{array}\right|=0;$$
$$A_{21}=(-1)^{2+1}\left|\begin{array}{ccc|ccc}1&0\\3&3\end{array}\right|=-3,\ A_{22}=(-1)^{2+2}\left|\begin{array}{ccc|ccc}1&0\\0&3\end{array}\right|=3,\ A_{23}=(-1)^{2+3}\left|\begin{array}{ccc|ccc}1&1\\0&3\end{array}\right|=-3;$$
$$A_{31}=(-1)^{3+1}\left|\begin{array}{ccc|ccc}1&0\\1&0\end{array}\right|=0,\ A_{32}=(-1)^{3+2}\left|\begin{array}{ccc|ccc}1&0\\0&0\end{array}\right|=0,\ A_{33}=(-1)^{3+3}\left|\begin{array}{ccc|ccc}1&1\\0&1\end{array}\right|=1.$$
то получаем:
$$\tilde A = \left(\begin{array}{rrr}3 & 0 & 0\\-3 & 3 & -3\\0 & 0 & 1\end{array}\right)$$
$$\Downarrow$$
$$\tilde A ^{\mathsf{T}} = \left(\begin{array}{rrr}3 & -3 & 0\\0 & 3 & 0\\0 & -3 & 1\end{array}\right)$$
$$\Downarrow$$
$$A^{-1}=\frac{1}{\det A}\tilde A ^{\mathsf{T}}=\left(\begin{array}{ccc}1 & -1 & 0\\ 0 & 1 & 0\\ 0 & -1 & 1/3\end{array}\right).$$
\section*{Задача 7}
{\bf Ответ:} $X=\left(\begin{array}{rrr}0 & 1 & 0\\0 & 2 & 0\\0 & 3 & 0\end{array}\right)$.
\\
\\
{\bf Решение.} Если матрица $A=\left(\begin{array}{rrr}1 & 1 & 1\\1 & 2 & 3\\1 & 4 & 9\end{array}\right)$ обратима, то $X=\left(\begin{array}{rrr}1 & 2 & 3\\2 & 4 & 6\\3 & 6 & 9\end{array}\right)A^{-1}$. Проверим существование $A^{-1}$:
\\
$$(A|E)=\left(\begin{array}{ccc|ccc}1 & 1 & 1 & 1 & 0 & 0\\1 & 2 & 3 & 0 & 1 & 0\\1 & 4 & 9 & 0 & 0 & 1\end{array}\right)\Leftrightarrow\left(\begin{array}{ccc|ccc}1 & 1 & 1 & 1 & 0 & 0\\0 & 1 & 2 & -1 & 1 & 0\\1 & 4 & 9 & 0 & 0 & 1\end{array}\right)\Leftrightarrow\left(\begin{array}{ccc|ccc}1 & 1 & 1 & 1 & 0 & 0\\0 & 1 & 2 & -1 & 1 & 0\\0 & 3 & 8 & -1 & 0 & 1\end{array}\right)\Leftrightarrow\left(\begin{array}{ccc|ccc}1 & 1 & 1 & 1 & 0 & 0\\0 & 1 & 2 & -1 & 1 & 0\\0 & 0 & 2 & 2 & -3 & 1\end{array}\right)$$
$$\Updownarrow$$
$$\left(\begin{array}{ccc|ccc}1 & 1 & 1 & 1 & 0 & 0\\0 & 1 & 0 & -3 & 4 & -1\\0 & 0 & 2 & 2 & -3 & 1\end{array}\right)\Leftrightarrow\left(\begin{array}{ccc|ccc}1 & 0 & 1 & 4 & -4 & 1\\0 & 1 & 0 & -3 & 4 & -1\\0 & 0 & 2 & 2 & -3 & 1\end{array}\right)\Leftrightarrow\left(\begin{array}{ccc|ccc}1 & 0 & 1 & 4 & -4 & 1\\0 & 1 & 0 & -3 & 4 & -1\\0 & 0 & 1 & 1 & -1.5 & 0.5\end{array}\right)$$
$$\Updownarrow$$
$$\left(\begin{array}{ccc|ccc}1 & 0 & 0 & 3 & -2.5 & 0.5\\0 & 1 & 0 & -3 & 4 & -1\\0 & 0 & 1 & 1 & -1.5 & 0.5\end{array}\right)=(E|A^{-1})$$
Таким образом существует $A^{-1}=\left(\begin{array}{ccc}3 & -2.5 & 0.5\\-3 & 4 & -1\\1 & -1.5 & 0.5\end{array}\right)$ и поэтому
$X=\left(\begin{array}{rrr}1 & 2 & 3\\2 & 4 & 6\\3 & 6 & 9\end{array}\right)\left(\begin{array}{ccc}3 & -2.5 & 0.5\\-3 & 4 & -1\\1 & -1.5 & 0.5\end{array}\right)$. Отсюда
$$x_{11}=3-6+3=0,\ x_{12}=-2.5+8-4.5=1,\ x_{13}=0.5-2+1.5=0;$$
$$x_{21}=6-12+6=0,\ x_{22}=-5+16-9=2,\ x_{23}=1-4+3=0;$$
$$x_{31}=9-18+9=0,\ x_{32}=-7.5+24-13.5=3,\ x_{33}=1.5-6+4.5=0$$
Таким образом $X=\left(\begin{array}{rrr}0 & 1 & 0\\0 & 2 & 0\\0 & 3 & 0\end{array}\right)$. При этом в силу обратимости $A$ других решений нет, т.к., если существует $X'\not = X$, такая что $X'A=\left(\begin{array}{rrr}1 & 2 & 3\\2 & 4 & 6\\3 & 6 & 9\end{array}\right)$, то, домножая справа на $A^{-1}$, получим $X'=\left(\begin{array}{rrr}1 & 2 & 3\\2 & 4 & 6\\3 & 6 & 9\end{array}\right)A^{-1}$, т.е. $X'=X$, что невозможно по предположению.

\end{document}